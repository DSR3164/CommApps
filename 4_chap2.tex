\chapter*{Практическая часть}
\addcontentsline{toc}{chapter}{Практическая часть}

В данной главе представлен анализ и описание реализации функционала получения и сохранения геолокации в Android‑приложении на базе Jetpack Compose. Основой приложения служит файл \texttt{Location.kt}, в котором последовательно выполняются следующие шаги:

\begin{enumerate}
    \item Запрос разрешений на доступ к геоданным (\texttt{ACCESS\_FINE\_LOCATION}, \texttt{ACCESS\_COARSE\_LOCATION}) с помощью \texttt{ActivityResultContracts.RequestMultiplePermissions}.
    \item Получение последней известной локации через \texttt{FusedLocationProviderClient}.
    \item Формирование JSON‑объекта с полями \texttt{latitude}, \texttt{longitude} и \texttt{timestamp}\cite{json_android}.
    \item Сохранение (дозапись) этого объекта в файл \texttt{location\_data.json} в директории \texttt{Environment.DIRECTORY\_DOCUMENTS}.
    \item Загрузка всей истории точек из файла и отображение в списке \texttt{LazyColumn} с карточками (\texttt{Card}).
\end{enumerate}

\subsection*{Основные компоненты и их взаимодействие}

\begin{itemize}
    \item \textbf{LocationActivity} (наследник \texttt{ComponentActivity}) задаёт тему приложения и запускает компоновку \texttt{LocationSaverScreen()}.
    \item \textbf{LocationSaverScreen}:
    \begin{itemize}
        \item \texttt{rememberLauncherForActivityResult} для запроса прав.
        \item \texttt{LaunchedEffect(Unit)} — начальная инициализация: загрузка существующих записей и запрос разрешений.
        \item Кнопка \texttt{Get Location} запускает логику проверок прав, асинхронный вызов \texttt{fusedLocationClient.lastLocation} и сохранение результата.
    \end{itemize}
    \item \textbf{LocationEntry} — дата‑класс для хранения одной записи.
    \item \textbf{LocationCard} — компоновка карточки с выводом временной метки и координат.
\end{itemize}

\subsection*{Пример кода}

\begin{lstlisting}[language=Kotlin, caption={Запись новой локации в файл}]
fusedLocationClient.lastLocation
    .addOnSuccessListener { location: Location? ->
        if (location != null) {
            val lat = location.latitude
            val lon = location.longitude
            val time = SimpleDateFormat(
                "yyyy-MM-dd HH:mm:ss",
                Locale.getDefault()
            ).format(Date())
            // Creating a JSON object
            val newObj = JSONObject().apply {
                put("latitude", lat)
                put("longitude", lon)
                put("timestamp", time)
            }
            // Reading an existing array or creating a new one
            val arr = try {
                JSONArray(file.readText())
            } catch (_: Exception) {
                JSONArray()
            }
            arr.put(newObj)
            // Writing it to an indented file
            FileWriter(file).use { it.write(arr.toString(2)) }
            statusText = "SAVED: \$time"
            entries = loadLocationsFromFile()
        } else {
            statusText = "Location is null"
        }
    }
    .addOnFailureListener { e ->
        statusText = "Location acquisition error: ${'$'}{e.message}"
    }
\end{lstlisting}

\subsection*{Тестирование и результаты}

Для проверки корректности работы выполнены тесты на эмуляторе и реальном устройстве:
\begin{itemize}
    \item Поведение при отказе в разрешениях: приложение повторно запрашивает права и отображает сообщение об ошибке.
    \item Сохранение нескольких точек: JSON‑файл формируется верно, добавляются новые записи, и они отображаются в списке.
    \item Обработка состояния, когда \texttt{lastLocation == null}: выводится соответствующий текст.
\end{itemize}

В результате получен стабильный модуль, который можно подключить к любому Android‑приложению для сбора и хранения геоданных пользователя.
