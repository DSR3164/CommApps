\chapter*{Теория}
\addcontentsline{toc}{chapter}{Теория}

При разработке мобильных приложений для платформы Android одной из ключевых задач является получение и обработка геопозиционных данных устройства. Геолокация на Android базируется на двух основных подходах:
\begin{itemize}
    \item \textbf{Использование системных провайдеров GPS и сети} через стандартный класс \texttt{android.location.LocationManager}. Данный подход обеспечивает детальное управление источниками (GPS, сотовая сеть, Wi-Fi), но требует сложной логики выбора оптимального провайдера, управления жизненным циклом запроса и значительных затрат ресурсов устройства.
    \item \textbf{Fused Location Provider API} из библиотеки Google Play Services (класс \texttt{FusedLocationProviderClient}). Абстрагирует детали разных провайдеров, автоматически выбирая наиболее точный и энергоэффективный источник в зависимости от условий и заданных параметров запроса.
\end{itemize}

\subsection*{Модель разрешений Android}
Начиная с версии Android 6.0 (API 23), политика безопасности платформы перешла на модель «рантайм-разрешений». Для доступа к геоданным необходимо запросить у пользователя одно из разрешений:
\begin{itemize}
    \item \texttt{ACCESS\_FINE\_LOCATION} — высокоточная локация (GPS, сеть).
    \item \texttt{ACCESS\_COARSE\_LOCATION} — приблизительная локация (сотовые вышки, Wi-Fi).
\end{itemize}
Приложение должно:
\begin{enumerate}
    \item Проверить наличие разрешений через \texttt{ActivityCompat.checkSelfPermission()}.
    \item При отсутствии — инициировать запрос с помощью \texttt{ActivityResultContracts.RequestMultiplePermissions}.
    \item Обрабатывать результат и корректно реагировать на отказ пользователя.
\end{enumerate}

\subsection*{Jetpack Compose и асинхронные операции}
Jetpack Compose предлагает декларативный подход к построению UI на Kotlin\cite{compose_runtime,kotlin_reference}. Для интеграции с API (такими как FusedLocationProviderClient\cite{fused_location_api}) в Compose используются:
\begin{itemize}
    \item \texttt{rememberLauncherForActivityResult} — для запуска запросов разрешений и получения их результата внутри композиции.
    \item \texttt{LaunchedEffect} — для выполнения побочных эффектов при старте или изменении состояний.
    \item \texttt{mutableStateOf} и \texttt{remember} — для хранения и отслеживания состояния UI (например, списка сохранённых точек или статуса операции).
\end{itemize}

\subsection*{Формат и организация хранения данных}
Для долговременного хранения точек геолокации используется файл в публичной директории \texttt{Environment.DIRECTORY\_DOCUMENTS}. Выбор \textbf{JSON} обоснован:
\begin{itemize}
    \item Читаемость и простота отладки.
    \item Широкая поддержка в стандартных библиотеках (\texttt{org.json.JSONArray}, \texttt{org.json.JSONObject}).
    \item Возможность форматирования с отступами для удобства ручного просмотра.
\end{itemize}
Структура файла:
\begin{verbatim}
[
  {
    "latitude": 55.7558,
    "longitude": 37.6173,
    "timestamp": "2025-05-19 14:23:10"
  },
  … 
]
\end{verbatim} 
Методы работы с файлом:
\begin{enumerate}
    \item \texttt{file.readText()} для чтения всего содержимого.
    \item Создание или парсинг \texttt{JSONArray} для добавления новых записей.
    \item Запись обратно с помощью \texttt{FileWriter} и форматированного вывода \texttt{toString(2)}.
\end{enumerate}

\subsection*{Отображение истории точек в UI}
Для визуализации сохранённых данных используется \texttt{LazyColumn} с элементами \texttt{Card}. Каждая карточка содержит:
\begin{itemize}
    \item Временную метку (\texttt{timestamp}) в формате \texttt{yyyy-MM-dd HH:mm:ss}.
    \item Координаты: широта (\texttt{latitude}) и долгота (\texttt{longitude}).
\end{itemize}
Преимущества такого решения:
\begin{itemize}
    \item Отложенная отрисовка элементов при большом числе записей.
    \item Возможность легко стилизовать и расширять карточку (иконки, кнопки, цветовые маркеры).
    \item Реактивное обновление списка при изменении состояния \texttt{entries}.
\end{itemize}

\subsection*{Безопасность и ограничения}
При работе с внешней памятью важно учитывать:
\begin{itemize}
    \item Разрешения на запись/чтение: \texttt{WRITE\_EXTERNAL\_STORAGE} (для старых версий) либо использование \texttt{Storage Access Framework}.
    \item Возможность отсутствия файловой системы (например, при отсутствии карты памяти).
    \item Обработка исключений для надёжного восстановления в случае ошибок I/O.
\end{itemize}

Таким образом, теоретическая база для получения, сохранения и отображения геопозиций в мобильном приложении объединяет в себе принципы работы Android API по локации, модель безопасности рантайм-разрешений, декларативный UI Jetpack Compose и практики организации JSON-хранилища.```
