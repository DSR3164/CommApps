\chapter*{Введение}
\addcontentsline{toc}{section}{ВВЕДЕНИЕ}

Актуальность темы исследования обусловлена широким распространением мобильных приложений для смартфонов и возрастающей ролью геолокационных сервисов в различных областях: навигации, сервиса «умного» дома, фитнес-трекеров и аналитики пользовательского поведения. Современные платформы Android предоставляют развитые API для получения координат устройства \cite{android_location,fused_location_api}(например, FusedLocationProviderClient из Google Play Services), однако эффективное и корректное сохранение этих данных требует учета аспектов безопасности, работы с файловой системой и пользовательского интерфейса.

Цель работы — разработать компонент мобильного приложения на базе Jetpack Compose, обеспечивающий получение последней известной геопозиции устройства и её сохранение в локальный JSON‑файл в публичной директории «Документы». Для достижения поставленной цели были решены следующие задачи:
\begin{enumerate}
    \item Анализ существующих средств Android API для получения геолокации (ACCESS\_FINE\_LOCATION, ACCESS\_COARSE\_LOCATION).
    \item Реализация запроса и обработки пользовательских разрешений на доступ к геоданным.
    \item Интеграция FusedLocationProviderClient в Jetpack Compose UI для асинхронного получения локации.
    \item Организация хранения собранных координат с временными метками в формате JSON в файле \texttt{location\_data.json}.
    \item Построение экрана на Compose, отображающего историю сохранённых точек в виде списка карточек.
    \item Тестирование корректности работы при различных состояниях разрешений и отсутствии данных.
\end{enumerate}

Методологической основой исследования послужили принципы реактивного программирования в Compose, рекомендации по работе с разрешениями Android и требования ГОСТ 7.32‑2017\cite{gost7322017} к оформлению отчёта о научно‑исследовательской работе.
