\chapter*{Заключение}
\addcontentsline{toc}{chapter}{Заключение}

В ходе выполнения работы была разработана и реализована компонента Android‑приложения на базе Jetpack Compose для получения последней известной геопозиции устройства и её сохранения в формате JSON в публичной директории «Документы». Основные результаты:
\begin{itemize}
    \item Успешно интегрирован FusedLocationProviderClient для получения координат.
    \item Реализована модель рантайм‑разрешений Android (\texttt{ACCESS\_FINE\_LOCATION}, \texttt{ACCESS\_COARSE\_LOCATION}) с корректной обработкой отказа пользователя.
    \item Организовано долговременное хранение точек через JSON‑файл (\texttt{location\_data.json}) с возможностью дозаписи и форматированием.
    \item Обеспечен наглядный вывод истории локаций в \texttt{LazyColumn} с реактивным обновлением списка.
\end{itemize}

Практическая ценность полученного решения заключается в том, что данный модуль может быть легко встроен в любые приложения, требующие сбора и визуализации геоданных пользователей. При дальнейшем развитии проекта возможны следующие направления:
\begin{enumerate}
    \item Добавление поддержки фонового отслеживания координат с использованием WorkManager или Foreground Service.
    \item Интеграция с внешними сервисами картографирования (Google Maps, OpenStreetMap) для визуализации точек на карте.
    \item Шифрование и безопасная передача данных на сервер для удалённого хранения и анализа.
\end{enumerate}
